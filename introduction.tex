Mental health is a critical component of public health, with profound implications for individuals and societies worldwide. According to the U.S. National Institute of Mental Health (NIMH), 22.8 percent of U.S. adults experienced mental illness in 2021, while the World Health Organization (WHO) reports that mental health disorders account for 30 percent of the global non-fatal disease burden, making them a leading cause of disability. These statistics highlight the urgent need for innovative solutions to address the growing mental health crisis, particularly in light of the global shortage of mental health professionals.

Large Language Models (LLMs) represent a groundbreaking advancement in artificial intelligence, characterized by their ability to process and generate human-like text at an unprecedented scale. Built on sophisticated architectures such as the Transformer, these models have demonstrated remarkable capabilities in understanding context, generating coherent responses, and performing complex language tasks. The evolution of LLMs, from early statistical models to modern systems like GPT-4 and LLaMA, has been driven by innovations in model architecture, training techniques, and computational efficiency.

The Transformer architecture, introduced by Vaswani et al. (2017), serves as the foundation for most contemporary LLMs. Its key innovation—the self-attention mechanism—enables models to dynamically weigh the importance of different words in a sequence, overcoming the limitations of earlier recurrent neural networks (RNNs) and long short-term memory (LSTM) models. Unlike RNNs, which process text sequentially and struggle with long-range dependencies, Transformers parallelize computations and capture global context more effectively. This architectural superiority has led to significant improvements in tasks requiring nuanced language understanding, such as sentiment analysis, dialogue generation, and mental health support applications.

This study aims to address this gap by providing the first detailed review of LLM applications in mental health care with four key areas:

1. Datasets, models, and training techniques used in mental health applications.

2. Applications and conditions targeted by LLMs, along with validation measures.

3. Discrepancies between current tools and their practical implementation in clinical settings.

4. Ethical and privacy considerations surrounding the use of LLMs in sensitive mental health contexts.