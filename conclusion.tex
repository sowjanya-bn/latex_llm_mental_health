In summary, while the mental health chatbot developed in this project provides a foundational step toward creating AI-driven support for individuals with depression, its current limitations, stemming from the use of a free platform, limited training data, and a lack of personalisation, restrict its effectiveness in real-world applications. Addressing these limitations by upgrading the platform, refining the model with more specialised data, and incorporating robust safety and personalisation features would significantly enhance the chatbot's ability to provide meaningful, context-sensitive, and safe support. With further development, this chatbot has the potential to become a valuable tool in mental health care, both for individual users seeking support and for therapists managing caseloads.
