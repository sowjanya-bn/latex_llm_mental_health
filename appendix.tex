\appendix

\section*{Appendix}
\addcontentsline{toc}{section}{Appendix}

\section{Datasets Used}
\label{appendix:datasets}

We used the following datasets to fine-tune and evaluate the chatbot models.

\begin{itemize}
    \item \textbf{Reddit Dataset 1: Depression Posts (TF-IDF Features)} \\
    A structured dataset containing Reddit posts from depression-related subreddits, with text features pre-processed using TF-IDF vectors. We specifically used the file \texttt{depression\_post\_features\_tfidf\_256.csv} available via Zenodo: \\
    \url{https://zenodo.org/records/3941387}
    
    \item \textbf{Reddit Dataset 2: Suicide Risk Posts} \\
    This dataset includes labeled Reddit posts from 500 users related to suicide risk with annotations, making it useful for identifying emotionally sensitive content. Hugging Face: \\
    \url{https://huggingface.co/datasets/m4faisal/RedditSuicide/commit/5e57cc7e2ebc706b98fbd853a2c52c10b582cd73}
    
    \item \textbf{Empathetic Dialogues Dataset} \\
    A conversational dataset containing 25,000+ short textual dialogues grounded in emotional situations, designed to train models to respond empathetically. Hugging Face: \\
    \url{https://huggingface.co/datasets/facebook/empathetic_dialogues}
\end{itemize}

These datasets were used to fine-tune different variants of the GPT-2 base model, enabling controlled comparisons of style, emotional engagement, and empathy.


\section{Code Used for Chatbot Implementation}
\label{appendix:code}

The complete implementation used for chatbot evaluation and generation is available on GitHub: \url{https://github.com/sowjanya-bn/latex_llm_mental_health/tree/main/code}
