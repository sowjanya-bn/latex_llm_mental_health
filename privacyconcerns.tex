The integration of large language models (LLMs) into mental health support systems introduces significant privacy challenges that must be addressed to ensure ethical and secure use. These challenges primarily involve data storage, usage, and access, necessitating robust measures to mitigate potential risks \cite{Volkmer2024-xw, Iwaya2022-ut}.
\subsection{Storage of Data}
Storing sensitive user data is a critical concern in LLM-driven mental health applications. Conversations often contain deeply personal information, making secure storage essential. Many AI mental health platforms use cloud-based storage solutions. which, if not properly protected, are susceptible to breaches. Regulations such as the General Data Protection Regulation (GDPR) in Europe impose strict guidelines on data storage, including requirements for anonymization and data minimization. However, compliance varies, and lapses can lead to significant privacy violations \cite{Iwaya2022-ut}.
\subsection{Use of Data}
The utilization of user data by LLM-based mental health platforms raises ethical and privacy concerns. These platforms often use user interactions to refine their algorithms through continuous learning, which can enhance personalization, but also introduce risks related to unintended data exploitation. There is concern that companies may repurpose anonymized data for research or commercial purposes without explicit user consent. Ethical frameworks suggest that informed consent should be obtained before user data is used beyond direct therapeutic engagement \cite{Guo2024-ba}.
\subsection{Access to Data}
Determining who has access to user data is a fundamental issue in LLM-based mental health applications. Although users expect confidentiality, service providers may need access to monitor AI performance and ensure the accuracy of mental health responses. However, concerns arise when third-party entities, such as external researchers, are granted access to anonymized datasets, raising the potential for data misuse. Mechanisms such as differential privacy and role-based access controls are being explored as potential solutions to mitigate these risks \cite{Iwaya2022-ut}.
\subsection{Ensuring Privacy in LLM-Based Mental Health Support}
Addressing privacy concerns in LLM-driven mental health support requires a multifaceted approach. Firstly, end-to-end encryption should be employed to protect user interactions from unauthorized access. Secondly, federated learning techniques, which allow AI models to be trained without transferring data to centralized servers, can help reduce privacy risks. In addition, regulatory compliance frameworks must be continuously updated to reflect evolving risks and ensure that AI-driven mental health applications adhere to the highest privacy standards. Finally, greater transparency with respect to data collection and processing, along with clear and informed user consent policies, are essential to maintain trust in AI-based mental health services \cite{Volkmer2024-xw,Iwaya2022-ut}.

